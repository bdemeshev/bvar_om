% !TEX TS-program = pdflatex
% !TEX encoding = UTF-8 Unicode

\documentclass[11pt]{article} % use larger type; default would be 10pt


%%% PAGE DIMENSIONS
\usepackage{geometry} % to change the page dimensions
\geometry{a4paper} % or letterpaper (US) or a5paper or\ldots .
% \geometry{margin=2in} % for example, change the margins to 2 inches all round
% \geometry{landscape} % set up the page for landscape
% read geometry.pdf for detailed page layout information


% \usepackage[parfill]{parskip} % Activate to begin paragraphs with an empty line rather than an indent

%%% PACKAGES
\usepackage{url}
\usepackage{soulutf8}
\usepackage{booktabs} % for much better looking tables
\usepackage{array} % for better arrays (eg matrices) in maths
\usepackage{paralist} % very flexible & customisable lists (eg. enumerate/itemize, etc.)
\usepackage{verbatim} % adds environment for commenting out blocks of text & for better verbatim
\usepackage{subfig} % make it possible to include more than one captioned figure/table in a single float
% These packages are all incorporated in the memoir class to one degree or another\ldots


%%% HEADERS & FOOTERS
%\usepackage{fancyhdr} % This should be set AFTER setting up the page geometry
%\pagestyle{fancy} % options: empty , plain , fancy




%%My Additional Packages
\usepackage{mathtext} % русские буквы в формулах
\usepackage[T2A]{fontenc} % внутренняя кодировка TeX
\usepackage[utf8]{inputenc} % кодировка исходного текста
% \usepackage{cmap} % русский поиск в pdf
\usepackage[english,russian]{babel} % локализация и переносы
\usepackage{amsmath} % Математические окружения AMS
\usepackage{amsfonts} % Шрифты AMS
\usepackage{amssymb} % Символы AMS
\usepackage{graphicx} % Вставить pdf- или png-файлы

\usepackage{euscript} % Красивый шрифт

\usepackage{longtable} % Длинные таблицы
\usepackage{multirow} % Слияние строк в таблице

\usepackage{indentfirst} % Отступ в первом абзаце.
\usepackage{pdflscape} %Переворачивает страницы, удобно для широких таблиц
\usepackage{footnote} %сноски в таблицах
\makesavenoteenv{tabular}
\makesavenoteenv{table}

%\usepackage[backend=biber, style=authoryear, citestyle=authoryear]{biblatex}
\usepackage[backend=biber, style=bwl-FU, citestyle=bwl-FU]{biblatex}


\addbibresource{bibliobase2.bib}

\usepackage{wrapfig} % Обтекание рисунков текстом

\usepackage{hyperref} % Гиперссылки

\DeclareMathOperator{\etr}{etr}
\DeclareMathOperator{\tr}{tr}
\DeclareMathOperator*{\argmax}{arg\,max}
\DeclareMathOperator*{\argmin}{arg\,min}
\DeclareMathOperator{\E}{\mathbb{E}}
\DeclareMathOperator{\diag}{diag}
\DeclareMathOperator{\Var}{\mathbb{V}\mathrm{ar}}
\DeclareMathOperator{\chol}{chol}
\newcommand{\cN}{\mathcal{N}}
\newcommand{\cIW}{\mathcal{IW}}
\newcommand{\lag}{\EuScript{L}}

\newcommand{\prior}{\underline}
\newcommand{\post}{\overline}

\let\vec\relax
\DeclareMathOperator{\vec}{vec}


%\usepackage[toc,page]{appendix}


%\renewcommand{\appendixtocname}{Приложения}
%\renewcommand{\appendixpagename}{Приложения}
%\renewcommand{\appendixname}{Приложение}
%%% END Article customizations

%%% The "real" document content comes below\ldots

\title{Guide for BVAR package}
%\date{} % Activate to display a given date or no date (if empty),
 % otherwise the current date is printed

\begin{document}
\pagenumbering{gobble}
\begin{center}
\begin{table}[h!]
\begin{tabular}{lp{10cm}}
\toprule
Название матрицы & Комментарий \\
\midrule
 \verb|all| & все данные, скаченные из EViews, содержат данные с 1992 г, включают столбец \verb|obs|, \verb|time| и \verb|time_y|\\
 \verb|df_sa| & то же, что и \verb|all|, но без \verb|obs| и \verb|time|\\
 \verb|df_final| & то же, что и \verb|df_sa|, но данные с 1995 г., часть переменных прологарифмирована\\ 
 \verb|df| & то же, что и \verb|df_final|, но с индексом \verb|t|, отражающим номер наблюдения\\
\midrule
 \verb|fit_set_2var| & показывает, какие переменные входят в \verb|fit_set| из двух переменных\\
 \verb|fit_set_3var| & показывает, какие переменные входят в \verb|fit_set| из трех переменных\\
 \verb|fit_set_info| & объединение \verb|fit_set_2var| и \verb|fit_set_3var| \\
 \verb|add_3| & список переменных, входящих в набор из трех переменных\\
 \verb|add_6| & список переменных, входящих в набор из шести переменных\\
 \verb|add_23| & список переменных, входящих в набор из 23 переменных\\
 \verb|var_set_info| & объединение таблиц \verb|add_3|, \verb|add_6| и \verb|add_23| \\
\midrule
 \verb|actual_obs| & таблица с данными, где все наблюдения стоят друг под другом\\
 \verb|deltas| & список $\delta_i$ для каждой переменной и вида модели для in-sample расчетов\\
 \verb|rwwn_list| & описание двух моделей для in-sample расчетов (RW и WN)\\
 \verb|rwwn_forecast_list| & обозначение моделей, для которых будут строиться in-sample прогнозы\\
\midrule
 \verb|rwwn_forecasts| & все in-sample прогнозы для всех переменных в каждый момент времени по RW и по WN\\
 \verb|rwwn_obs| & объединение \verb|rwwn_forecasts| c фактическими наблюдениями и квадратом ошибки\\ 
 \verb|msfe0_all| & среднеквадратичная ошибка по каждой переменной по RW и WN\\
 \verb|rwwn_wlist| & описание оцененных RW и WN моделей. \\
 \verb|msfe0| & среднеквадратичная ошибка по каждой переменной для каждой переменной в соответствии с тем классом временного ряда, который был выбран тестом на (не)стационарность\\
\midrule
 \verb|var_list| & описание всех частотных VAR-моделей, подлежащих оценке\\
 \verb|var_forecast_list| & список VAR моделей, по которым строятся прогнозы\\
 \verb|var_forecasts| & in-sample прогнозы по VAR-моделям (для каждого периода времени, сортировано по моделям)\\
 \verb|var_obs| & то же, что и \verb|var_forecasts| плюс фактические значения и квадрат ошибки\\
 \verb|msfe_Inf| & msfe in-sample прогнозов VAR, сортировано по переменным\\
\midrule
 \verb|var_wlist| & подробное описание оцененных VAR моделей\\
 \verb|msfe_Inf_info| & объединение \verb|msfe_Inf| с информацией о самой модели \\
 \verb|msfe_0_Inf| & объединение \verb|msfe_Inf_info| с msfe по RW или WN и соотношение MSFE\\
 \verb|fit_inf_table| & подсчет \verb|fit| для каждого \verb|fit_set|, \verb|var_set| и количества лагов \\
 \verb|bvar_list| & список всех BVAR моделей (отличаются набором переменных, количеством лагов и гиперпараметрами\\
\bottomrule
\end{tabular}
\end{table}
\end{center}

\newpage

\begin{center}
\begin{table}[h!]
\begin{tabular}{lp{10cm}}
\midrule

\verb|bvar_forecast_list|& список прогнозов по BVAR моделям\\
\verb|bvar_forecasts| &прогнозы in-sample по всем переменным в каждый момент времени\\ 
\verb|bvar_obs|& то же, что и \verb|bvar_forecasts|, но с фактическими наблюдениями и квадратом ошибки\\
\verb|msfe_lam|& среднеквадратичная ошибка для каждой переменной и каждой BVAR-модели, в которой она участвует\\
\verb|bvar_wlist|& полная таблица с описание BVAR моделей\\
\midrule
\verb|msfe_lam_info|& объединение \verb|msfe_lam| с описанием каждой модели\\
\verb|msfe_0_lam| &объединение \verb|msfe_lam_info| c msfe по RW или WN и соотношение MSFE\\
\verb|fit_lam_table|& считает \verb|fit| для каждой возможной модели, для каждого \verb|var_set| и \verb|fit_set|\\
\verb|fit_goal|& содержит \verb|fit| для VAR-модели из 3 переменных для разного количества лагов и для разных \verb|fit_set|\\
\verb|fit_lam_table|& сопоставляет с \verb|fit_lam_table| c \verb|fit_goal| и рассчитывает разницу \verb|fit|-ов.\\
\midrule
\verb|best_lambda|&для каждой модели и номера лагов выбирает такое сочетание гиперпараметров, при котором отличие \verb|fit| в BVAR наименьшее\\
\verb|bvar_out_list|& описание всех моделей BVAR, для которых будет строиться прогноз out-of-sample\\
\verb|bvar_out_forecast_list|& расширенный список оцененных BVAR, подсчитанных по схеме rolling window \\
\verb|bvar_out_forecasts|& out-of-sample прогнозы для всех переменных, для всех BVAR моделей для каждого момента времени\\
\verb|bvar_out_obs| & совмещает \verb|bvar_out_forecasts| с фактическими значениями и информацией о модели, считает квадрат ошибки\\
\midrule
\verb|omsfe_bvar_table|& содержит out-of-sample MSFE для каждой переменной для каждого набора данных, количества лагов, прогнозного окна и \verb|fit_set|\\
\verb|rwwn_var_unique_wlist|& содержит список моделей (VAR,RW,WN) с которыми происходит сравнение прогнозной способности BVAR\\
\verb|rwwn_var_out_list|& полная информация по всем RW,WN и VAR моделям, по которым строится прогноз out-of-sample\\ 
\verb|rwwn_var_out_forecast_list|& список всех прогнозных моделей и максимальных прогнозных окон по ним. Модели, использующие наблюдения, в конце выборки имеют меньшее прогнозное окно, чем более ранние за отсутствием наблюдений, с которыми прогноз будет сравниваться. \\
\verb|rwwn_var_out_forecasts|& все out-of-sample прогнозы по VAR,RW и WN моделям \\
\verb|rwwn_var_out_obs|& сопоставляет \verb|rwwn_var_out_forecasts| с фактическими наблюдениями и с информацией о типе модели, также считает квадрат ошибки\\
\verb|omsfe_rwwn_var_table|& выдает MSFE по каждой переменной для моделей VAR,RW и WN для каждого набора переменных, количества лагов, прогнозного окна\\
\verb|omsfe_selected_rwwn|& выбирает MSFE по RW и WN для каждой переменной в соответствии с тестом на (не)стационарность\\
\midrule  
\end{tabular}
\end{table}
\end{center}
\newpage
\begin{center}
\begin{table}[h!]
\begin{tabular}{lp{10cm}}
\midrule 
\verb|omsfe_bvar_table|& MSFE прогнозов по каждой переменной в зависимости от \verb|fit_set|, количества лагов и горизонта прогнозирования\\
\verb|omsfe_var_banbura|& MSFE по VAR - моделям только для переменных, входящих в \verb|desired_fit_set|\\
\verb|omsfe_var_banbura|& MSFE по BVAR - моделям только для переменных, входящих в \verb|desired_fit_set|\\
\verb|omsfe_rwwb_banbura|& MSFE по RW и WN - моделям только для переменных, входящих в \verb|desired_fit_set|\\
\midrule 
\verb|var_bvar_omsfe_banbura_table|& MSFE по VAR и BVAR, нормированные на MSFE по RW и WN, для переменных, входящих в \verb|desired_fit_set|. \\ 
\verb|all_rmsfe_wide| & Перегрупировка \verb|var_bvar_omsfe_banbura_table|\\
\verb|some_rmsfe_wide|& То же, что и \verb|all_rmsfe_wide|, но для выбранных прогнозных окон\\
\verb|create_best_var_list|& выбор наилучшего лага для VAR по критерию Шварца\\
\bottomrule


\end{tabular}
\end{table}
\end{center}


\end{document}